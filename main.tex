\documentclass[conference]{IEEEtran}
\IEEEoverridecommandlockouts
% The preceding line is only needed to identify funding in the first footnote. If that is unneeded, please comment it out.



%\include{jdeters}


\usepackage{cite}
\usepackage{amsmath,amssymb,amsfonts}
\usepackage{algorithmic}
\usepackage{graphicx}
\usepackage{textcomp}
\usepackage{xcolor}
\usepackage{booktabs, multirow} % for borders and merged ranges
\usepackage{soul}% for underlines
\usepackage{comment}
\usepackage{wrapfig}
\usepackage{capt-of}
\usepackage{url}
\include{jdeters}
\def\BibTeX{{\rm B\kern-.05em{\sc i\kern-.025em b}\kern-.08em
    T\kern-.1667em\lower.7ex\hbox{E}\kern-.125emX}}


%%
%% end of the preamble, start of the body of the document source.
\begin{document}

%%
%% The "title" command has an optional parameter,
%% allowing the author to define a "short title" to be used in page headers.
\title{Feature-Oriented FSMs for FGPAs}

\title{Feature-Oriented FSMs for FPGAs
\thanks{Supported under NSF CISE award CNS-1763503 Performant Architecturally Diverse Systems via Aspect-oriented Programming}
}

\author{\IEEEauthorblockN{Justin Deters, Peyton Gozon, Max Camp--Oberhauser, and Ron K. Cytron}
\IEEEauthorblockA{\textit{Department of Computer Science and Engineering} \\
\textit{Washington University}\\
St. Louis, MO, USA \\
\{jdeters, peyton.gozon, c.max, cytron\}@wustl.edu}

}
\long\def\Cache#1{#1}

\maketitle

%%
%% The abstract is a short summary of the work to be presented in the
%% article.
\begin{abstract}
Reconfigurable architectures enable the economical deployment of application-specific designs, and RISC-V has emerged as an open and flexible processor for supporting such applications.  Generally, some portions of a design may be executed on the RISC-V processor with other portions present in circuitry surrounding the processor.  Notably, a particular application may need only a subset of the architecture and microarchitecture of a full-featured RISC-V implementation.

In this paper we consider a feature-oriented approach for specifying finite-state machines, which form the basis of cache controllers (and other components) for RISC-V implementations, and which are commonly found in hardware designs. Using a library we constructed for Chisel, developers can apply features at will, with the resulting machine containing only the circuitry needed to support the desired features.  

Our library offers two constructs for building features. The first, inspired by aspect-oriented programming, applies incremental changes to the states and edges of a finite-state machine to alter and customize its behavior in response to features of interest.  The second construct couples the behavior of separate finite machines into a single machine that processes its inputs simultaneously.  We illustrate each construct separately using a vending machine and the game of Nim, respectively.
%%
\Cache{We then apply the constructs in concert to generate a coherent, 2-way multiprocessor cache from a much simpler specification.}

Our approach offers significant leverage in supporting both the number and size of the generated designs.  We present results from synthesis that show the size of the design endpoints compared with the much smaller size of their specification.
\end{abstract}


\maketitle

\section{Introduction}

Projects such as RocketChip~\cite{chisel:riscv} and RISC-V~Mini~\cite{Kim:22} allow customization of RISC-V~\cite{riscvwebsite} processors that can then be deployed economically on FPGAs.  These resulting systems can offer better power and price performance than general-purpose processors while achieving a price point well below generation of an ASIC (application-specific integrated circuit).

While some portions of the RISC-V characterization are easily included or excluded at will, based on a given application's needs, components such as cache, branch-prediction circuitry, and superscalar support are much more difficult to weave in or excise from a given characterization.

In the RISC-V implementations cited above, components of the processor such as its cache and bus protocols rely on finite-state machines~(FSMs) to control the sequencing of the associated logic.   Hardware implementations commonly rely on FSMs at the core of their design, due to their simplicity and efficiency.

In this paper, we consider FSM controllers in which  features should be included based an application's needs.  In a system with $n$~such independent features, there are $2^n$~possible endpoint designs that could be generated.  Maintaining those as distinct projects is unwieldy and inefficient.  Instead, we use techniques from aspect-oriented programming to weave features into a design on demand.  The resulting code base is significantly smaller, while maintaining the ability to generate any endpoint on demand.   The resulting characterization must still undergo synthesis, but the system in its feature-factored form is easier to maintain.  Moreover, testing of all feature combinations can be easily automated.

Hardware-generation languages such as Chisel~\cite{chisel:article} allow a designer to write a program whose \emph{execution} generates the hardware design. The program can be authored using para\-digms that promote efficiency, reuse, rigorous testing, and clarity of expression. Our work builds on the hardware-generation language Chisel~\cite{chisel:book}, which is in turn built on Scala~\cite{scala-overview-tech-report}. Chisel is a Scala-embedded domain-specific language with libraries that generate Verilog~\cite{verilog} when a Chisel program is executed.

Chisel has proven itself robust in pedagogy and industry, serving as the basis for courses in digital logic~\cite{vlsicourse} and serving as a platform for describing full-featured RISC-V architectures~\cite{chisel:riscv}.  

In this paper, we consider a feature-oriented programming (FOP) approach to generating hardware, specifically \emph{finite-state machines} (FSMs).   Our contributions are as follows:
\begin{itemize}
    \item In Section~\ref{sec:vend}, we describe an approach to formulating features in FSMs based on aspect-oriented programming (AOP).  We apply this idea to a featureful vending machine in Section~\ref{sec:ccut} and present results on the automatically generated FSMs with specific feature sets in Section~\ref{sec:vendresults}.
    \item In Section~\ref{sec:nim} we describe a generative \emph{cross-product} technique (due to Harel~\cite{HAREL1987231}) that composes larger FSMs from smaller, behavior-specific machines, illustrated using the game of Nim.  Results from its application to variations of Nim are shown in Section~\ref{sec:nimresults}.
    \Cache{
    \item We combine both of the above ideas in Section~\ref{sec:cache} to the generation of an SIMD cache, similar to one currently in use by~AMD.  The parallelism of the cache is generated via the cross-product technique and a coherence scheme is imposed as a feature using the AOP technique.}
    \item We present results in Section~\ref{sec:results} that show our approach's significant leverage toward generating large, complex FSMs from relatively smaller and simpler ones. 
    \item The code we have developed for this work integrates with Chisel and is available via \texttt{github}.
\end{itemize}
The FSMs generated using our approach enjoy benefits similar to feature-oriented software systems:
\begin{itemize}
    \item Smaller footprints are obtained by including only those features of interest.  The resulting hardware designs have fewer states and less logic, supporting only the features of interest.
    \item Logic related to inactive features is completely absent from the resulting designs. Since the clock rate of a hardware system is determined in part by the amount of logic that must complete in a clock cycle, less logic could lead to faster clock rates.
\end{itemize}


\section{Prior work and motivation}\label{sec:prior}

We review here the concepts and prior work upon which our work is based.
\subsection{Aspect-oriented programming}

Our work builds on a programming paradigm available in the software community that efficiently supports the expression and application of \emph{cross-cutting} concerns in a software system.  Aspect-Oriented Programming~(AOP)~\cite{gregor:97} and its realization in systems such as AspectJ~\cite{aspectj} allow developers to express ideas that affect multiple components of a system in support of a common idea or feature. 

Aspects have been applied to finite-state machines for sequence diagrams in a modeling (UML) setting~\cite{aspectsUML}.  Our use of aspects extends that work by taking advantage of types in Scala to formulate advice and guide FSM modifications. 

\subsection{Benefits of feature-oriented programming}\label{sec:benefop}

Our work concerns the generation of FSMs that incorporate desirable subsets of features.  We show that FSMs can be specified much more efficiently using this approach.  Moreover, the FSMs ultimately generated by the Chisel toolchain contain only those features of interest, requiring fewer resources than would a full-featured, monolithic specification whose undesirable features are disabled.

This approach has also been attractive for managing product lines in which features evolve over time, affecting not only the base system but also other features~\cite{10.1145/2897695.2897701}. 

An example from the software world of the benefits obtainable from an FOP approach to a featureful system concerns the CORBA~\cite{CORBA:00} Event Channel~\cite{CORBAService:02a}. The standard implementation is \emph{monolithic}, offering all possible features in all allowable combinations.  A decomposition of the Event Channel in terms of its features has demonstrated that useful subsets of those features use significantly fewer resources when formulated generatively~\cite{Pratap:04}.   


\section{Generative FSM specifications}

We next illustrate our two FOP constructs for generating complex FSMs.  The first uses aspect-oriented advice to incorporate features selectively into an FSM.  The second builds a \emph{cross-product} FSM from the synchronous simulation of smaller FSMs.

\subsection{Feature introduction via AOP}\label{sec:vend}

As an example of a featureful hardware design, we consider an FSM implementation of a vending machine.   A state in our design carries the necessary (Scala) \emph{traits} to represent its role in the machine's operation:  the funds inserted and the potential products dispensed.  Our generative approach described below offers the following advantages over a monolithic design:
\begin{itemize}
    \item The design itself is simpler and clearer when described using FOP.  A monolithic design that tangles all features can be realized, but the resulting FSM does not readily make the features apparent.  Moreover, the work to create that monolithic implementation is tedious and error-prone.
    \item Modification of the FSM is greatly simplified.  For example, introduction of a new value of coinage automatically creates the necessary additional states and transitions.
    \item Scala traits allow elegant expression of an application's behaviors in support of FOP hardware design.
\end{itemize}
For this example and the results we present, the features of interest for a vending machine are as follows:
\begin{description}[\IEEEsetlabelwidth{Insufficient Funds}]
    \item[Add Currency] introduces a value of coinage.
    \item[Dispense Product] introduces a vendible item and its price.
    \item[Print Funds] causes the machine to display the total funds after each state change.
    \item[Insufficient Funds] introduces a prompt to advise the consumer to insert more funds to buy a particular item.
    \item[Change Return] introduces a button (input to the FSM) that causes the machine to return unspent funds.
    \item[Peanut Warning] requests confirmation of purchase for items that contain peanuts.
    \item[Buy More] allows the consumer to continue purchasing items if funds remain in the machine.  The \textbf{Change Return} feature, if present, allows the consumer to request return of the remaining funds.
\end{description}
The dependencies of these features are shown in Figure~\ref{fig:vmDependencies}, but this graph is not needed for construction:  the advice for a given feature is applicable only when its associated \emph{join points} exist in the FSM.  As is typical with aspect-oriented approaches, all advice is presented to a \emph{weaver} (our runtime library for Chisel), and the aspects are continually applied until no changes occur.

For example, the advice for \textbf{Add Currency} of coinage $k$ specifies that for any state representing that $n$ cents have been inserted, a state representing $n+k$ cents must exist, with a transition from state~$n$ to state~$n+k$ based on the insertion of coinage $k$.   This advice fails to terminate if not capped by some upper bound on funds, which could be related to the most expensive product sold. For example, Figure~\ref{fig:vend1} shows a machine that
\begin{itemize}
    \item Accepts only 5 cent coinage
    \item Accepts up to 15 cents
    \item Vends peanuts that cost 10 cents
\end{itemize}
The FSM is automatically generated by the advice for \textbf{Add Currency} and \textbf{Dispense Product}.  Continuing with this example, consider the \textbf{Buy More} feature, intended to incentivize consumers to spend more money.  This feature causes the machine to retain remaining funds after a purchase to encourage subsequent purchases.  Without this feature, the machine in Figure~\ref{fig:vend1} would return 5~cents if 15~cents are used to purchase peanuts costing 10~cents.  With the feature, the 5~cents of remaining value would be held by the machine for subsequent purchases.  The advice for this feature modifies \emph{every} purchase to move to a state representing currently held funds.  In Section~\ref{sec:ccut} we discuss application of other features to this FSM.

A monolithic approach requires designers to specify all states and transitions for each feature subset, which is tedious and error-prone. With our approach, designers can verify the correctness of much smaller designs and then obtain much larger generative designs that are correct by their construction.

In terms of leverage, consider an FSM for which there are $n$ orthogonal and independent features.  A valid system could thus be written or generated with or without each of those $n$ features.  This leads to $2^{n}$ feature-specific implementations.  While it is unlikely that each of those implementations would find an application, the ability to generate \emph{any} of them automatically offers significant leverage.

\begin{figure*}
    \centering
    \includegraphics[width=0.5\textwidth]{figures/vend1.pdf}
    \caption{FSM for a vending machine that accepts 5 cent coins and dispenses peanuts that cost 10 cents.}
    \label{fig:vend1}
\end{figure*}

\begin{figure}
    \centering
    \includegraphics[width=0.4\linewidth]{figures/VendingMachine.pdf}
    \caption{Dependencies between vending machine features.}
    \label{fig:vmDependencies}
\end{figure}

\subsection{Generating FSMs via cross-product composition}\label{sec:nim}

Nim~\cite{enwiki:1102668015} is a broad class of impartial mathematical strategy games, which traditionally involve multiple heaps of tokens (\emph{e.g.}, sticks) and two or more alternating players. The current player removes an allowable number of sticks from a subset of the heaps. The winner is usually defined as the player taking the last token. In a \emph{mis\`{e}re} version of the game, that player would lose.

In contrast with the usual monolithic solution, even the most basic game of Nim can be regarded as the composition or simultaneous operation of two simpler machines:   one that represents only the allowable subtractions of tokens in a heap (such as the 5-token heap shown in Figure~\ref{fig:nimHeapFSM}(a)) and one that represents only the alternation of players (such as the two-player alternation shown in Figure~\ref{fig:nimHeapFSM}(b)).  Transitions not shown in those machines are errors, such as Player~A taking two consecutive turns.  

Following is our feature decomposition of Nim:
\begin{description}[\IEEEsetlabelwidth{Num Players}]
    \item[Heap Bounds] encodes the initial and winning number of tokens for each heap.
    \item[Legal Moves] encodes permissible combinations of adding or removing tokens from each heap.
    \item[Num Players] specifies how many players take turns in the game.
    \item[Win Type] specifies whether the game is mis\`{e}re play or normal play.
\end{description}
The dependencies of these features is shown in Figure~\ref{fig:nimDependencies}.  

\begin{figure}
    \centering
    \includegraphics[width=0.4\linewidth]{figures/NimFeatures.pdf}
    \caption{Dependencies between Nim features.}
    \label{fig:nimDependencies}
\end{figure}


The game is won when both conditions are met:
\begin{itemize}
    \item Players alternate correctly, as in Figure~\ref{fig:nimHeapFSM}(b).  For example, the sequence ABA leads to an accept, but the sequence AAB cannot.
    \item All tokens have been taken, for example using the sequence 212 in Figure~\ref{fig:nimHeapFSM}(a).
\end{itemize}
% Formally, in the theory of regular languages, acceptable inputs for our problem are in the \emph{concatenation} of the languages of these two machines.   Given the sequence ABA and 212, each accepted by its respective machine, the concatenation ABA212 is a string in the concatenation of the two languages.  While this correctly represents the sequence of inputs that causes somebody to win, the timing of actions associated with transitions in the two machines does not coincide properly.  For example, it is not possible in the concatenation to determine \emph{who} won the game when the second machine moves to its accepting states, where zero tokens remain. 

\begin{figure}
    \centering
    \begin{tabular}{cc}
    \includegraphics[width=0.35\linewidth]{figures/nimexample/heapFSM.pdf} &
    \includegraphics[width=0.3\linewidth]{figures/nimexample/playerFSM.pdf} \\
    (a) & (b)
    \end{tabular}
    \caption{(a) FSM for a 5-token heap that allows one or two tokens to be removed in a turn; (b) FSM specifying alternation of Players~A and~B.}
    \label{fig:nimHeapFSM} 
\end{figure}

% We prefer to process the inputs more naturally as (A,2), (B,1), and (A,2), so that the player making the move and the number of subtracted tokens are processed in lock-step.  In this way, any actions associated with each move will be properly taken upon that move.  Similarly, if the game had 3~heaps in play instead of a single heap, we want all heaps to be appropriately modified by each turn.  Concatenation would focus on one heap until it is exhausted before moving to the next heap.   Finally, consider that the first machine accepts~ABAB which would be inappropriate if the second machine accepts~212, since only~3 moves are made.  The lock-step execution we require is over \emph{same-length} inputs for each machine.


Using a construction technique due to Harel~\cite{HAREL1987231} and well documented in Ptolomy~\cite{EdwardLee}, we obtain the basic game of Nim shown in Figure~\ref{fig:nimFSM}.  That algorithm simulates the simultaneous, lock-step execution of the machines shown in Figures~\ref{fig:nimHeapFSM}(a) and~(b).

In terms of leverage, consider the cross-product generation of an FSM from two identical FSMs each of size $m$ (states+transitions).  The resulting machine's worst-case size is~$O(m^{2})$.  An $n$-way cross product generates a machine of size $O(m^{n})$, where $m$ is viewed as a constant here.  The structures we can generate with this approach are (in the limit) exponential in the size of their specifications.

\begin{figure}
    \centering
    \includegraphics[width=0.25\textwidth]{figures/nimexample/nimFSM.pdf}
    \caption{The resulting Nim finite-state machine.  The edge transitions are labeled with the player who acts to take the specified number of tokens. The state is labeled with the player who just completed a turn and the number of remaining tokens.}
    \label{fig:nimFSM}
\end{figure}

\section{Formalism for cross-cutting features}\label{sec:formal}\label{sec:ccut}
We begin with an FSM $M$, typically defined as follows: 
\[M = (Q, \Sigma, \delta, q_0, F)\]where $Q$ is a set of states, $\Sigma$ is a set of tokens, $\delta$ is the transition function, $q_0$ is the start state, and $F$ is the set of accepting states.  A state is typically denoted by an upper-case letter;  lower-case letters denote tokens and strings.  The symbol $\lambda$ denotes the empty string.  When an FSM is drawn as a graph, the start state receives an edge with no sources, and an accepting state is drawn with two concentric circles. 

The formalism presented here are implemented in Chisel as a library we call Foam, described in Section~\ref{sec:foam}.  The examples and results presented in this paper were created using Foam.  


We follow~\cite{aspectsUML} in the treatment of aspects for FSMs.  Essentially, a state is like a method and a transition between states is like a method call.  The usual forms of before, after, and around advice are available (\emph{cf}. AspectJ~\cite{AspectJ:01}).   A cross-cutting feature is implemented using advice that modifies an FSM's behavior before, after, or during a transition between states.

As described below, a feature is comprised of \emph{advice} applied to \emph{pointcuts} of an FSM, which can formally change the language of the machine.  More broadly and usefully, the advice can affect actions taken by the machine as its inputs are processed.  

\paragraph{Pointcuts} These specify \emph{where} advice should be applied in a targeted FSM.   The generative nature of Chisel eliminates any need for new syntax to express pointcuts.  Instead, we can select states or symbols using simple set quantifiers and predicates, written in Chisel/Scala and executed along with the rest of the Chisel code that generates a circuit.  For example, the \textbf{Print Funds} feature can be generated in a vending-machine FSM through \emph{after} advice applied to any token that adds value to the machine.  Such properties are supported nicely in Scala using \emph{traits}.  To implement this feature, the base code likely requires refactoring to include the value trait.  However, the effort is worthwhile because the refactoring and associated advice make the resulting product both clearer and more easily able to exclude or include actions taken at different inputs.

In AOP terminology, a pointcut yields a set of \emph{join points} at which advice is applied.  A join point associated with the above example would be a single token~``5'', such at the one between state ``10'' and ``15'', at which the value increases in the machine by ``5'' cents.  Because this is an \emph{after} pointcut, the join point has context that includes the state~``15'' that follows the token~``5'', as shown in Figure~\ref{fig:vend1}. 

\begin{figure}
    \centering
    \includegraphics[width=0.5\textwidth]{figures/vend2.pdf}
    \caption{The resulting FSM of the application of Print Funds to the FSM from Figure \ref{fig:vend1}.}
    \label{fig:applyadvice}
\end{figure}

\paragraph{Advice} This specifies \emph{what} changes to the FSM should be applied at a join point. In the \textbf{Print Funds} feature, a new state is inserted following each token~``5'' in the FSM. The exact print statement generated by the advice is determined by the context contained within the join point. For example, \texttt{Print \textcent15} is generated because the state~``15'' follows the token~``5'' discussed earlier. Furthermore, to prevent infinite application of features, the advice for \textbf{Print Funds} checks to see if the context in the join point is already a printing state. If so, no advice is applied. Like pointcuts, advice is written in Chisel/Scala. 

Not only can advice insert new states, but it can also insert new symbols as well. Consider the \textbf{Peanut Warning} feature as an example. The pointcut is predicated upon a dispense state having a ``peanut'' trait. Because this is a \emph{before} pointcut, each join point has context that includes transition information that targets the state. In Figure~\ref{fig:applyadvice2} this is $10 \xrightarrow{\text{Peanut}}$ and $15 \xrightarrow{\text{Peanut}}$. The advice will insert a new ``Contains Nuts!''~state and ``Accept''~token for each of the join points. It also inserts a new transition on the ``Reject''~token whose destination is determined by the context contained in the join point.

\begin{figure}
    \centering
    \includegraphics[width=0.5\textwidth]{figures/vend3.pdf}
    \caption{The resulting FSM of the application of Peanut Warning to the FSM from Figure \ref{fig:vend1}.}
    \label{fig:applyadvice2}
\end{figure}


\section{Aspect-Oriented Finite-State Machine Library}\label{sec:foam}
We have implemented an aspect-oriented finite-state machine library in Scala that we call Foam. Here we discuss the library as well as code generation.

While Foam is not realized as a domain-specific language, we have modeled the interface after the well-established aspect-oriented extension to Java, AspectJ~\cite{aspectj}. The intention is to provide aspect-oriented practitioners a familiar interface for interacting with the FSMs. Because the library is implemented in ordinary Scala, we have access to and can utilize the full power of its type system. 

We provide a series of extendable base classes to represent FSMs. The library builds pointcuts, applies aspects, and performs the cross-product construction. Like AspectJ, our library provides access via reflection to a join point and its context.  For the examples presented here, the library allows efficient and clear expression of features.  Figure~\ref{lst:PrintFunds} shows the implementation of the \textbf{Print Funds} feature in Foam for the FSM vending machine.

\begin{figure*}
    \centering
    \begin{lstlisting}[language = Scala]
class PrintFunds extends Aspect[NFA] {
  def apply(nfa: NFA) = {

    val tokenPointcut = Pointcutter[Token, Coin](nfa.alphabet, token => token match {
      case t: Coin => true
      case _ => false
    })

    AfterToken[Coin](tokenPointcut, nfa)((thisJoinpoint: TokenJoinpoint[Coin], thisNFA: NFA) => {
      var value = thisJoinpoint.out.asInstanceOf[ValueState].value
      thisJoinpoint.out match {
        case s: PrinterState if (s.action == "Funds:" + value.toString) => (None, thisNFA)
        case _ => (Some((PrinterState("Funds:" + value, value, false), Lambda)), thisNFA)
      }
    })
  }
}
\end{lstlisting}
    \caption{The implementation of the Print Funds feature in Foam.}
    \label{lst:PrintFunds}
\end{figure*}

\subsection{Code Generation}
The library currently supports emitting DOT and Verilog code. Code generation is decoupled from the creation of the FSMs, as the library generates code based off the internal data structures, not the Scala code itself. DOT and Verilog code are generated by Graphviz4S~\cite{Ldpe2G:19} and Chisel, respectively. 

\section{Case studies and results}\label{sec:results}
Here we present three case studies to demonstrate the generative ability of our framework: a Vending Machine, the game of Nim, and SIMD cache coherence.

\subsection{Vending Machine}\label{sec:vendresults}
We implemented all the features from Section \ref{sec:vend} in our library. The resulting FSMs were then emitted as Verilog. The Verilog was synthesized on a xc7a35tcpg236-1 FPGA using Vivado 2022.1. Below we report the number of generated states, transitions in the FSM, and the space in LUTs that the FSM took up in on the FPGA.

Figure~\ref{fig:vmData} shows the results for different endpoints generated by our library. For our tests, we held the currency threshold at 100. Every machine contains 5, 10, and 25 value coins; and 4 products of value 25, 50, 75, and 100. This is captured by \emph{None}. In the first set of results, each feature is shown by itself. Even single features can greatly increase the components in the FSM. The \textbf{Buy More} feature (denoted B) by itself more than doubles the number of states and transitions. This impressive leverage is further exemplified when combining features. 

In these cases the number of states increase by 2.5x in the simplest endpoint up to 4.6x in the most complex, and the transitions by 2.5x and 5x respectively. Recall, this is in a relatively simple vending machine that can only accept up to 100 units of value. Simply doubling the amount of accepted value to 200 creates a machine with 284 states (10.5x increase over the base) and 3113 transitions (16x increase over the base). However, this is accomplished in our library with relatively few lines of code. The largest feature in terms of code is \textbf{Peanut Warning}, which is implemented in just 39 lines.

Despite the growing number of generated states and transitions as the features increase in Figure~\ref{fig:vmData}, the resulting hardware resources, in this case Look Up Tables (LUTs), used by the FSM are relatively modest. This is because hardware synthesis tools can represent states using a linear encoding scheme.  For an FSM with $n$ states, each state takes only $O(\log{n})$ space when encoded as an integer.  However, the \emph{specification} of that circuit to a synthesis tool must be expressed state-by-state.  If the number of transitions per state is bounded by a constant, then the specification (\textit{e.g.}, lines of Verilog) takes $O(n)$ space.  Our generative approach to hardware FSMs opens hardware designers to implementing much larger FSMs than are currently sustainable with a hand-coded approach.

\begin{figure}\small
    \centering
\begin{tabular}{lrrrr}\toprule
Features &States &Transitions &LUTs \\\midrule
None &27 &208 &24 \\
P &47 &368 &38 \\
I &47 &368 &46 \\
C &48 &423 &70 \\
W &33 &320 &60 \\
B &57 &448 &60 \\
PI &67 &528 &73 \\
PICB &118 &1053 &186 \\
PICW &94 &1023 &160 \\
PICWB &124 &1353 &220 \\
\bottomrule
\end{tabular}
    \caption{Number of generated states, transitions, and LUTs depending on selected features. The features are as follows: P = Print Funds, I = Insufficient Funds, C = Change Return, W = Peanut Warning, and B = Buy More.}
    \label{fig:vmData}
\end{figure}

\subsection{Nim}\label{sec:nimresults}

We implemented all the features described in Section~\ref{sec:nim} and studied these variations of Nim: traditional Nim and circle Nim. All results assume each game is mis\`{e}re play with at least one heap and one player. We further assume that the players alternate in a round-robin fashion. The details of each variation are described below, along with the number of generated states and transitions within the created FSMs. 

\paragraph{Traditional Nim}
The classical game of Nim contains two players and three heaps, though there are no restrictions on the number of players or quantity of heaps. The defining characteristic of traditional Nim is that, each turn, the current player may only remove sticks from a single heap of their choosing, and must take between 1 and all the remaining sticks within that heap. The game ends when all heaps contain zero sticks. 

Figure~\ref{tab:traditionalNim} shows the results for different endpoints of the traditional game of Nim, varying both the quantity of heaps, the number of sticks within each heap, and the number of players. In the simplest variation, containing only a single heap with three sticks and one player, a total of six states and 30 transitions were generated. Juxtapose this with the most complex variation, containing three heaps -- with three, four, and five sticks, respectively -- and four players: this variation contains 419 states and 20950 transitions. With minimal changes to the specification of the game, this represents a nearly 70x increase in the number of states and 698x increase in the number of transitions. For a fixed number of players, the number of transitions experiences growth by an order of magnitude moving from 3 to 4 to 5 heaps. 

\begin{figure}\small\centering
\begin{tabular}{rrrrrrr}\toprule
\multicolumn{3}{c}{Heaps} &Players &States &Transitions \\\cmidrule{1-6}
3 &4 &5 & & & \\\midrule
\checkmark & & &1 &6 &30 \\
\checkmark & & &2 &9 &63 \\
\checkmark & & &4 &11 &88 \\
\checkmark &\checkmark & &1 &22 &198 \\
\checkmark &\checkmark & &2 &39 &624 \\
\checkmark &\checkmark & &4 &60 &1800 \\
\checkmark & \checkmark &\checkmark &1 &112 &1708 \\
\checkmark & \checkmark &\checkmark &2 &235 &6110 \\
\checkmark & \checkmark &\checkmark &4 &419 &20950 \\
\bottomrule
\end{tabular}
\caption{Number of generated states and transitions by selected features for traditional Nim.}\label{tab:traditionalNim}
\end{figure}

% \paragraph{The Subtraction Game}
% This game is a single-heap variation of Nim, usually played by a small group of players. In our analysis, we fix the starting number of sticks to 21 each game. Each turn, the current player removes between one and a set number of sticks from the heap, with the game ending when the final stick is removed. 

% Figure~\ref{tab:subtractionNim} shows the results for different endpoints of the subtraction game, varying the number of sticks permissible to take each turn and the number of players. In the simplest variation, with there only being one player allowed to take one stick each turn, the resulting FSM contains 24 states and 72 transitions. By comparison, the most complex variation considered has four players and allows the player to take 1, 2, or 3 sticks each turn, with the corresponding FSM having 83 states and 1162 transitions. This marks an almost 3.5x increase in the number of states and 16x increase in the number of transitions. Empirically, the primary driver for complexity in these variations is increasing the number of heaps, which the subtraction game lacks; however, the ability to generate such a breadth of variations for a single game type with minimal code changes is a huge lever. 

% \begin{figure}
% \small\centering
% \begin{tabular}{cccrrrr}\toprule
% \multicolumn{3}{c}{Allowed} & & & & \\
% \multicolumn{3}{c}{Moves} &Players &States &Transitions \\\cmidrule{1-6}
% -1 &-2 &-3 & & & \\\midrule
% \checkmark & & &1 &24 &72 \\
% \checkmark & & &2 &24 &96 \\
% \checkmark & & &4 &24 &144 \\
% \checkmark &\checkmark & &1 &24 &96 \\
% \checkmark &\checkmark & &2 &45 &270 \\
% \checkmark &\checkmark & &4 &81 &810 \\
% \checkmark &\checkmark &\checkmark &1 &24 &120 \\
% \checkmark &\checkmark &\checkmark &2 &45 &360 \\
% \checkmark &\checkmark &\checkmark &4 &83 &1162 \\
% \bottomrule
% \end{tabular}
% \caption{Number of generated states and transitions by selected features for the Subtraction game. Every game is played with a single starting heap of 21 sticks.}\label{tab:subtractionNim}
% \end{figure}

\paragraph{Circle Nim}
Circle Nim is given its name due to the layout of its heaps: the heaps are placed around a circle, with their adjacencies affecting gameplay. It is traditionally played with a finite number of heaps, each containing a single stick, and two players. Each turn, the current player is allowed to take the sticks from between one and a pre-set number of consecutive heaps. We allow the player to take from between one and three consecutive heaps. 

Figure~\ref{tab:circleNim} shows the results for different endpoints of circle Nim, varying both the number of players and the number of heaps. In the simplest variation, with three heaps and a single player, 10 states and 110 transitions are generated. In the most complex variation, with nine heaps and four players, 1429 states and 157190 transitions are generated, representing a 143x and 1420x increase, respectively. Only two numbers were changed in code to realize this exponential increase in output complexity. 

\begin{figure}
\small\centering
\begin{tabular}{rrrrr}\toprule
Heaps &Players &States &Transitions \\\midrule
3 &1 &10 &110 \\
3 &2 &15 &255 \\
3 &4 &17 &340 \\
6 &1 &66 &1320 \\
6 &2 &113 &4294 \\
6 &4 &150 &11100 \\
9 &1 &514 &14906 \\
9 &2 &951 &53256 \\
9 &4 &1429 &157190 \\
\bottomrule
\end{tabular}
\caption{Number of generated states and transitions by selected features for Circle Nim. Each heap contained one stick; players could take from between one and three consecutive heaps.}\label{tab:circleNim}
\end{figure}

\subsection{SIMD Caches}\label{sec:cache}
Ultimately, the goal of this work is to make the design of complex hardware easier through generation. Here, we will demonstrate how we can combine our techniques with ``off-the-shelf'' components to generate SIMD Caches that are coherent with each other. While we recognize that this is purely a pedagogical example, it is modeled off the real-world RDNA Architecture from AMD~\cite{AMD:19}.

The RDNA architecture packages two SIMD execution units into a single unit called a \emph{workgroup} of processors. Each workgroup shares L0 cache between each SIMD unit. This cache is kept coherent through serialization of the execution when conflicts are detected. Two workgroups are then packaged together into a \emph{Dual Compute Unit} (DCU). Currently, the two L0s within a DCU are kept coherent via software. 

Suppose that we wanted to model a similar cache system ourselves, but instead of handling data conflicts between the two SIMD units in a workgroup processor, we use the MSI cache coherence protocol~\cite{Good83}. For this case study, we have selected the ready-made cache from RISC-V Mini~\cite{Kim:22}. As-is, this cache, shown in Figure \ref{fig:cacheBefore}, is directly connected to main memory in a single core system and does not have any coherence protocol. Using a feature-oriented approach, we can retrofit this cache with advice that implements the MSI protocol, shown in Figure \ref{fig:cacheAfter}. In order to model the interactions between the two now coherent caches, we can cross-product the two. This results in an FSM with 729 states and 2,461,104 transitions. Hardware designers could use this sort of modeling for correctness verification, however that is beyond the scope of this paper.

\begin{figure}
    \centering
    \includegraphics[width=0.8\linewidth]{figures/cacheFSM.pdf}
    \caption{The cache FSM from the RISC-V Mini.}
    \label{fig:cacheBefore}
\end{figure}

\begin{figure}
    \centering
    \includegraphics[width=0.98\linewidth]{figures/cacheFSM2.pdf}
    \caption{The cache FSM from RISC-V Mini with the MSI protocol.}
    \label{fig:cacheAfter}
\end{figure}

Imagine that in the next iteration of the architecture, the designers wanted to use hardware coherence between all the SIMD units in a DCU. Instead of having to start the model completely from scratch, hardware designers can simply instantiate two more caches into the system. Furthermore, rather than being locked into a single cache coherence protocol, the hardware designers may want to choose the MESI protocol~\cite{Papamarcos:84}, for their next iteration. With our approach, this protocol can just be applied instead of the MSI protocol. 

\section{Summary and future work}

We have described an FOP approach to constructing complex finite-state machines from much simpler ones.   We have illustrated our ideas using two pedagogical examples and one real-world setting, namely an SIMD cache.   We have presented an algorithm that creates a cross-product FSM, which simulates the lock-step simultaneous execution of its two input FSMs.  Our results confirm that this FOP approach provides significant leverage in terms of the size of the generated products, as compared with the relatively smaller effort of authoring the individual features.

We are currently working to create a fully feature-oriented cache using our system. Hardware caches are ripe for feature-oriented design as they contain many orthogonal features. For example, if we wanted to build a cache model even closer to the RDNA architecture, the original cache FSM would need to be write-through, 4-way set associative, and utilize an LRU replacement policy. Instead of forcing hardware designers into choosing an initial design and refactoring, write policy, allocation policy, replacement policy, and associativity could all be selectable features of the microarchitecture.

% Beyond the cache, it is tempting to extend this concept of feature-oriented hardware design to the chip as a whole. Hennessy and Patterson's original motivation behind the RISC-V architecture was to create an open and \emph{modular} instruction set. RISC-V applications span small, application-specific processors (such as the controllers for Western Digital storage products~\cite{WD:22} and Google's Titan M2 security chip~\cite{kleidermacher:21}) to high-performance processors (such at the SiFive P550~\cite{SiFive:22}). Instead of independently implementing each of these endpoints of RISC-V and the microarchitecture, we posit that an FOP implementation of RISC-V would be far more useful. In that world, hardware designers could swap and customize each of the ISA and microarchitectural features as needed.  For hardware designers, this could introduce a marketplace for feature exchange---something that the software community has enjoyed for over a half century.

%%
%% The next two lines define the bibliography style to be used, and
%% the bibliography file.
%\clearpage
\bibliographystyle{plain}
\bibliography{acmart,bibdbase,networks}

\end{document}
\endinput
%%
%% End of file `sample-sigplan.tex'.
